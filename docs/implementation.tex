\section{Implementation}

\subsection{Starting Point}

The first step to implementing SF is to determine what part of the Linux kernel needs to be modified, specifically, which system call(s). Since SF is involved with limiting the opening of files, the starting point for determining which system call(s) are used is to look at what happens when a file is opened.

To examine what happens when a file is opened, \texttt{strace} is used on commands such as \texttt{cat}, for example, to see what system calls are called to read from a file. While there are many system calls that are output, by looking at what parameters are passed to these system calls, it can be narrowed down to the \texttt{open} system call.

Now that the system call has been determined, the source code of the kernel is looked through to see what kernel functions the \texttt{open} system call uses. Other than the check at the beginning of the \texttt{open} system call to check if it is a large file, \texttt{open} is just made up of the function \texttt{do\_sys\_open}. Thus, the starting point is narrowed down to \texttt{do\_sys\_open}.

\subsection{Extended Attributes}

Since the design of the system uses the extended attributes of a file, implementing that functionality is a first.

To implement the extended attributes, the source code is, once again, looked through to find the source of extended attribute functionality. The system calls \texttt{getxattr} and \texttt{setxattr} are discovered in the \texttt{fs/xattr.c} file for getting and setting the extended attributes, respectively. Looking at the body of the \texttt{getxattr} and \texttt{setaxttr}, they only call the kernel functions \texttt{path\_getxattr} and \texttt{path\_setxattr}, respectively. Thus, to implement the extended attributes functionality, those kernel functions have to be copied into the file of \texttt{do\_sys\_open}, \texttt{fs/open.c}, as they are not defined in a header file. Since functions from a different file are being copied to \texttt{fs/open.c} and it is not clear what header files are needed for the functions, all of the \texttt{\#include} statements are copied over to avoid any potential problems.

\subsection{Directory Restriction}

Before modifying files with the extended attributes, the files that are to be affected by SF have to be restricted. To restrict the files, the function \texttt{in\_restricted\_path} is made, which gets called in \texttt{do\_sys\_open}. 

In \texttt{in\_restricted\_path} a dentry struct is obtained from the file pointer parameter. The dentry struct has a parent dentry field which can be iterated to obtain the file path. That is then compared to the restricted file path to determine whether or not the file should be considered by SF.

By having the files restricted to a certain directory, the likelihood of affecting any important files needed by the kernel is reduced. 

\subsection{Permanent Lockout}

For the main logic of SF, the extended attributes functionality is used to set and get three attributes of the files: what time, in epoch time, the first open time (FOT), the total time it has been opened for (TOT), and the open count (OC). These attributes are called \texttt{user.open\_time\_first}, \texttt{user.open\_time\_total}, and \texttt{user.open\_count\_attr}, respectively.

If a file is opened for the first time, then the OC is 0 so it gets set to 1, which involves setting the \texttt{user.open\_count\_attr} attribute. As outlined in the design, since the OC is 0 when it is opened, then the FOT is set, so the \texttt{user.open\_time\_first} is set to the current time.

When a file is opened, SF gets the \texttt{user.open\_time\_total} attribute. As the design states, SF checks if it is greater than an open limit. If it is, the file is refused the open. Instead of opening the file, it will return an error. Specifically, it returns the EPERM error which is `operation not permitted' so the calling process with be able to handle it appropriately. 

To set the values of extended attributes, the function \texttt{path\_setxattr} is used. It normally would take a user space string of file path of the file as one of the parameters. For SF, this function has been modified because the file paths passed to calls for \texttt{path\_setxattr} are made in kernel space. To correct for this, in the body of \texttt{path\_setxattr}, there are lines removed that would move the information from user space memory to kernel memory. As well, the file path string parameter is changed to a path struct which allowed for simplifying \texttt{path\_setxattr}. To simplify calls to \texttt{path\_sexattr} a little more, a simple wrapper function \texttt{long\_sexattr} is made to call \texttt{path\_sexattr} with all the default flags to avoid repetition.

For getting the extended attributes, the exact same method is employed, but \texttt{path\_getaxttr} is the main function used. It is modified in the same way as \texttt{path\_getxattr} to compensate for the user space pointer issue. A wrapper function \texttt{long\_getxattr} is also created to simply the calls to \texttt{path\_getxattr}.

As mentioned in the design, these values also have to be set during the closing of the file. The function \texttt{on\_close} is created to handle the logic of closing a file. It operates as laid out in the design, so it handles setting the TOT in the \texttt{user.open\_time\_total} attribute and decrementing OC by one.

With this system, once a file has an OT greater than the set open limit, it is always set. The file will have to be deleted because it can now no longer be opened.

\subsection{Daily Resets}

To deal with files being permanently locked out once their total open time is too great, SF resets files when the file is opened on a new day. i. e., if today a file reached its open time limit, opening the file after midnight will set its open time back to 0 so it can be opened again for that day.

To check if the open time of a file is from a previous day, functions and structs from \texttt{<linux/time.h>} are used. The \texttt{tm} struct holds the day number of 0 to 365 and the year number, among other things. Using \texttt{time\_to\_tm}, an epoch time stamp can be converted to a \texttt{tm} struct. This is used to turn a file's first open time to a \texttt{tm} to see what day and year it was last opened on. If the year or day number is less than the current year or day number, then SF will allow the open to proceed because it must be a new day.

Before allowing the opening to proceed, however, SF checks if the file is already open by checking the open count in the \texttt{user.open\_count\_attr} attribute. If the count is greater than 0, meaning another file has it open, the first open time is set to midnight of the current day. The reason being, if the file is already open, then it must have been open over midnight, which is the start of a new day. Therefore, it has to start calculating its open time from that point forward. Using the function \texttt{to\_midnight}, which utilizes other members in \texttt{tm} structs, midnight in epoch time can be calculated and then set.

\subsection{Issues}
\subsubsection*{User Space Pointers}
Originally, when \texttt{path\_setaxttr} or \texttt{path\_getxattr} got called, both of them would return an error 14. After extensive internet searching, it is observed that an error 14 in the kernel is a `bad address' error. With that in mind, it must be some error with one of the pointers being passed in. The first thought for the cause of the error is that the errors are coming from the \texttt{char*} and \texttt{void*} parameters. After many attempts at passing different values in for the \texttt{char*} and \texttt{void*} parameters, the functions would always return error 14. 

With every possible way to pass values in for the mentioned parameters, the strategy for isolating the problem is changed. To locate the problem, \texttt{printk} statements are placed in the functions everywhere an error is being checked for from the return of a function call. These \texttt{printk} statements help locate which call or calls inside \texttt{path\_setaxttr} and \texttt{path\_getxattr} that are returning the error 14. Through doing this, the errors are tracked to being produced by functions such as \texttt{strncpy\_from\_user}. The value being passed to this function is the value in the \texttt{char*} \texttt{\_\_user} \texttt{pathname} parameter. 

Discovering where the error is originating from, the cause is apparent. The function is trying to copy information from user space to kernel space. More specifically, the information stored in memory from the user space pointer is trying to be copied to memory in kernel space. Since the 
\subsubsection*{Relative Paths}