\section{Implementation}
\subsection*{Starting Point}
The first step to implementing SF is to determine what part of the Linux kernel needs to be modified, specifically, which system call(s). Since SF is involved with limiting the opening of files, the starting point for determining which system call(s) are used is to look at what happens when a file is opened.

To examine what happens when a file is opened, the tool \texttt{strace} can be used. While in the terminal, calling on \texttt{strace} on a command will output all the system calls that occur from the running of a program. \texttt{strace} is used on commands such as \texttt{cat}, for example, to see what system calls are called for the opening of a file. While there are many system calls that are output, by looking at what parameters are used being passed to these system calls, it can be narrowed down to being the \texttt{open} system call.

Now that the system call that has been determined, the source code of the kernel is looked through to see what kernel functions the \texttt{open} system call calls. Other than the check at the beginning of the \texttt{open} system call to check if it is a large file, \texttt{open} is just made up of the function \texttt{do_sys_open}. 