\section{Introduction}

\subsection{Purpose}

The project is meant to limit the amount of time a file can be kept open each day by refusing subsequent attempts to open the file. The idea is a simplified version of giving ownership of files to specific programs. The two ideas share the following characteristics: a process can be allowed or denied the ability to open a particular file, and relevant information is stored in file's extended attributes.

If a file is open longer than a set time, all the processes with it already open are allowed to keep it open. If any process attempts another open on the same file, then a \textit{Operation Not Permitted} error is returned.

As soon as the computer clock passes midnight then any previous total open time on files is ignored, and the total open time is set back to 0. Thus, whenever a new day starts, files can be opened again for their allotted amount of time.

\subsection{Design}

To store all the data required to determine whether or not a file can be opened, extended attributes are used. Extended attributes are, essentially, key-value pairs which can be stored on individual files; thus, all the information for one file is stored with that file.

To track how long a file is open, a value is stored as the \textit{Total Open Time} (TOT) (i.e. the time for which any already closed file was open for). A value also stores the current number of opens, \textit{Open Count} (OC), for a file (i.e. the number of times the file has been opened without being closed). Finally, a value stores the time at which the first open was performed, \textit{First Open} (FO), (i.e. the time at which an open was performed and OC was zero).

